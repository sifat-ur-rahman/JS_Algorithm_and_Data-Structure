         জাভাস্ক্রিপ্ট অ্যালগরিদম ও ডাটা স্ট্রাকচার

    বিগ ওহ (O) নোটেশনঃ
=> এটি কমপ্লেক্সিটি বুঝানোর জন্য ইউজ করা হয় ।
=> একটা অ্যালগরিদমের টাইল/স্পেস কমপ্লেক্সিটির আপার বাউন্ড বুঝাতে ব্যবহার করা হয় ৷ 
   এটা মূলত অ্যালগরিদমের সবচেয়ে খারাপ কেস(ওরস্ট কেস) হিসেব করতে ব্যবহার করা হয়৷
=> বিগ ‘ওহ’(o) এর সাথে আরো দুইটা নোটেশন দেখে থাকবেন ‘ওমেগা’(Ω) ও
   'থেটা’(θ) নোটেশন নামে



=> একটা মাত্র অ্যারে/কালেকশনের উপর একটা লুপ চালানো হলে।
    সম্ভবত O(n) হতে পারে।


=> দুইটা ভিন্ন কালেকশন/অ্যায়ের উপর দুইটা আলাদা আলাদা লুপ চালানো হলে।
    সম্ভবত O(n+m) হতে পারে।


=> অবজেক্ট {} এ কমন কিছু অপারেশনের বিগ 'ওহ' হবে।
    ১। ইন্সার্ট করতেঃ O(1)
    ২। রিমুভ করতে: O(1)
    ৩। সার্চ করতে: O(n)
    ৪। অ্যাক্সেস করতে: O(1)


=> অ্যারেতে [] কমন কিছু অপারেশনের বিগ ‘ওহ’ হবে 
    ১। ইন্সার্ট করলেঃ
        1. push(): O(1)
        2. Sunshift(): O(n)
    ২। রিমুভ করলেঃ
        1. pop(): O(1)
        2. shift(): O(n)
    ৩। সার্চিং: O(n)
    ৪। অ্যাক্সেসঃ O (1)

